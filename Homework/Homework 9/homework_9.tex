% (c) 2005 Lukasz Grzegorz Maciak

% this is a template I use for my exam papers

\documentclass[a4paper,10pt]{article}

\setlength{\textheight}{10in}
\setlength{\textwidth}{6.5in}
\setlength{\topmargin}{-0.125in}
\setlength{\oddsidemargin}{-.2in}
\setlength{\evensidemargin}{-.2in}
\setlength{\headsep}{0.2in}
\setlength{\footskip}{0pt}

%\usepackage{fontspec}
\usepackage{amsmath}
%\usepackage{unicode-math}
%\setmathfont{texgyrepagella-math.otf}
%\setmathfont{latinmodern-math.otf}
%\setmathfont{xits-math.otf}
%\setmathfont{Asana-Math.otf}
\usepackage{fancyhdr}
\usepackage{enumitem}
%\usepackage{amssymb}
%\usepackage{bm}
%\usepackage{graphicx}
%\usepackage{stmaryrd}
\usepackage{hyperref}
\hypersetup{
	colorlinks   = true, %Colours links instead of ugly boxes
	urlcolor     = blue, %Colour for external hyperlinks
	linkcolor    = red, %Colour of internal links
	citecolor   = red %Colour of citations
}
\usepackage{graphicx}
\usepackage{listings}

\pagestyle{fancy}
%\renewcommand{\footrulewidth}{0.4pt}

\lhead{\textbf{Name:} \rule{5.5cm}{0.5pt}}
%\chead{M.Number: \rule{2cm}{0.5pt}}
\chead{\textbf{Homework 9}}
\rhead{\textbf{Fundamentals of CS \& Programming}}
\fancyfoot{}
%\newcommand{\bigbrk}{\vspace*{2in}}
%\newcommand{\smallbrk}{\vspace*{.3in}}

\begin{document}

\section*{Running times of loops}

Write a small Python script to measure the running time of loops. First, define a number \( n \).  
Before each loop define a counter variable and set it to zero. In each step of the loop increment the counter by 1.  
After the loop has finished, print the counter. Implement the following loops:

\begin{enumerate}[label=\arabic*)]
    \item \textbf{Loop A.} Write a \texttt{while} loop that starts with the loop variable being equal to \( n \) and runs until the loop variable is larger than 0. Within the loop integer divide the loop variable by two.
    
    \item \textbf{Loop B.} Write a single \texttt{for} loop from 0 to \( n - 1 \).
    
    \item \textbf{Loop C.} Write a nested loop with Loop B as the outer loop and Loop A as the inner loop.
    
    \item \textbf{Loop D.} Write a nested loop with Loop B as the outer loop and Loop B as the inner loop, i.e., a nested loop with two \texttt{for} loops.
    
    \item \textbf{Loop E.} Write a nested loop with Loop D as the outer loop and Loop A as the inner loop, i.e., Loop A is enclosed in two \texttt{for} loops.
    
    \item \textbf{Loop F.} Write a nested loop with three \texttt{for} loops.
\end{enumerate}

\bigskip
\noindent
\textbf{Instructions:}  
In your code, write comments after printing the results and discuss the running times you have obtained.

\newpage
\mbox{} 

\newpage
\mbox{} 

\end{document}




