% (c) 2005 Lukasz Grzegorz Maciak

% this is a template I use for my exam papers

\documentclass[a4paper,10pt]{article}

\setlength{\textheight}{10in}
\setlength{\textwidth}{6.5in}
\setlength{\topmargin}{-0.125in}
\setlength{\oddsidemargin}{-.2in}
\setlength{\evensidemargin}{-.2in}
\setlength{\headsep}{0.2in}
\setlength{\footskip}{0pt}

%\usepackage{fontspec}
\usepackage{amsmath}
%\usepackage{unicode-math}
%\setmathfont{texgyrepagella-math.otf}
%\setmathfont{latinmodern-math.otf}
%\setmathfont{xits-math.otf}
%\setmathfont{Asana-Math.otf}
\usepackage{fancyhdr}
\usepackage{enumitem}
%\usepackage{amssymb}
%\usepackage{bm}
%\usepackage{graphicx}
%\usepackage{stmaryrd}
\usepackage{hyperref}
\hypersetup{
	colorlinks   = true, %Colours links instead of ugly boxes
	urlcolor     = blue, %Colour for external hyperlinks
	linkcolor    = red, %Colour of internal links
	citecolor   = red %Colour of citations
}
\usepackage{graphicx}
\usepackage{listings}

\pagestyle{fancy}
%\renewcommand{\footrulewidth}{0.4pt}

\lhead{\textbf{Name:} \rule{5.5cm}{0.5pt}}
%\chead{M.Number: \rule{2cm}{0.5pt}}
\chead{\textbf{Homework 2}}
\rhead{\textbf{Fundamentals of CS \& Programming}}
\fancyfoot{}
%\newcommand{\bigbrk}{\vspace*{2in}}
%\newcommand{\smallbrk}{\vspace*{.3in}}

\begin{document}

\section*{1. Propositions}
Which of the following sentences are propositions? What are the truth values of those that are propositions?

\begin{enumerate}
  \item Boston is the capital of Massachusetts.
  \item Berlin is the capital of Austria.
  \item $2 + 3 = 5$.
  \item $5 + 7 = 10$.
  \item $x + 2 = 11$.
  \item Answer this question.
  \item $x + y = y + x$ for every pair of real numbers $x$ and $y$.
\end{enumerate}

\newpage
\mbox{} 

\newpage
\mbox{}

\section*{2. Propositions as English sentences}
Let $p$, $q$, and $r$ be the propositions:
\[
\begin{aligned}
p&:\ \text{You have the flu.} \\
q&:\ \text{You miss the final examination.} \\
r&:\ \text{You pass the course.}
\end{aligned}
\]

Express each of the following propositions as an English sentence.
\begin{enumerate}
  \item $p \to q$
  \item $\bar{q} \leftrightarrow r$
  \item $q \to \bar{r}$
  \item $p \lor q \lor r$
  \item $(p \to \bar{r}) \lor (q \to \bar{r})$
  \item $(p \land q) \lor (\bar{q} \land r)$
\end{enumerate}

\newpage
\mbox{} 

\newpage
\mbox{}

\section*{3. Truth tables}
Construct a truth table for each of the following compound propositions.

\begin{enumerate}
  \item $(p \lor \bar{q}) \to q$
  \item $p \oplus p$
  \item $(p \oplus q) \lor (p \oplus \bar{q})$
  \item $p \to \bar{q}$
  \item $(p \leftrightarrow q) \lor (\bar{p} \leftrightarrow q)$
\end{enumerate}

\newpage
\mbox{} 

\newpage
\mbox{}

\section*{4. Logical puzzle}
This puzzle can be solved by translating statements into logical expressions and reasoning from these expressions using truth tables:

Four friends have been identified as suspects for an unauthorized access into a computer system. They have made statements to the investigating authorities.

\begin{itemize}
  \item Alice said ``Carlos did it.''
  \item John said ``I did not do it.''
  \item Carlos said ``Diana did it.''
  \item Diana said ``Carlos lied when he said that I did it.''
\end{itemize}

\begin{enumerate}
  \item If the authorities also knew that exactly one of the four suspects is telling the truth, who did it? Explain your reasoning.
  \item If the authorities also know that exactly one is lying, who did it? Explain your reasoning.
\end{enumerate}


\newpage
\mbox{} 

\newpage
\mbox{}

\section*{5. Tautology with truth table}
Show that each of the following implications is a tautology by using truth tables.

\begin{enumerate}
  \item $(\bar{p} \land (p \lor q)) \to q$
  \item $((p \to q) \land (q \to r)) \to (p \to r)$
  \item $(p \land (p \to q)) \to q$
  \item $((p \lor q) \land (p \to r) \land (q \to r)) \to r$
\end{enumerate}

\newpage
\mbox{} 

\newpage
\mbox{}

\section*{6. Tautology with laws of logic}
Show that each implication from the previous example is a tautology without using truth tables.

\newpage
\mbox{} 

\newpage
\mbox{}

\section*{7. Logical equivalence}
Show that $\overline{p \oplus q}$ and $p \leftrightarrow q$ are logically equivalent.



\newpage
\mbox{} 

\newpage
\mbox{} 
\end{document}




